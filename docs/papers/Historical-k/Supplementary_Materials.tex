% Supplementary Materials for K(t) Index Manuscript
% Supporting Information for "A Multi-Harmonic Index of Global Civilizational Coherence"
% Nature Sustainability Submission

\documentclass[11pt,letterpaper]{article}

% Packages
\usepackage[utf8]{inputenc}
\usepackage[T1]{fontenc}
\usepackage{amsmath,amssymb}
\usepackage{graphicx}
\usepackage{booktabs}
\usepackage{longtable}
\usepackage{natbib}
\usepackage[margin=1in]{geometry}
\usepackage{hyperref}
\usepackage{caption}
\usepackage{subcaption}

% Hyperref setup
\hypersetup{
    colorlinks=true,
    linkcolor=blue,
    citecolor=blue,
    urlcolor=blue
}

% Title
\title{\textbf{Supplementary Information} \\
       A Multi-Harmonic Index of Global Civilizational Coherence: \\
       Historical Reconstruction and Validation, 1810--2020}

\author{Tristan Stoltz \\
        Luminous Dynamics, Richardson, TX, USA \\
        \texttt{tristan.stoltz@luminousdynamics.org}}

\date{\today}

\begin{document}

\maketitle

\tableofcontents
\newpage

% ============================================================================
% SECTION S1: DETAILED SEVEN HARMONIES DEFINITIONS
% ============================================================================

\section{Detailed Seven Harmonies Definitions}
\label{sec:si-harmonies}

This section provides comprehensive definitions of each of the seven harmonies comprising the Historical K(t) Index, including theoretical foundations, proxy selection rationale, and measurement methodology.

\subsection{H₁: Resonant Coherence (Governance Quality)}

\textbf{Conceptual Definition}: The capacity of governance institutions to coordinate collective action through legitimate, responsive, and effective decision-making processes.

\textbf{Theoretical Foundation}: Drawing on political science literature (Acemoglu \& Robinson 2012), we conceptualize governance quality as encompassing both democratic participation (input legitimacy) and administrative effectiveness (output legitimacy). High resonant coherence implies that governance institutions successfully aggregate diverse preferences into coherent policy while maintaining accountability.

\textbf{Proxy Variables} (V-Dem Database, World Bank Governance Indicators):
\begin{itemize}
    \item Electoral democracy index (0--1 scale)
    \item Participatory democracy index (0--1 scale)
    \item Government effectiveness (percentile rank)
    \item Rule of law (percentile rank)
    \item Control of corruption (percentile rank)
    \item Voice and accountability (percentile rank)
\end{itemize}

\textbf{Vision-Proxy Gap}: These metrics capture institutional \textit{capacity} for coordination (free elections, functional bureaucracies) but not \textit{quality} of coordination (whether decisions serve collective wellbeing, whether trust exists between citizens and institutions). A surveillance state with high administrative effectiveness scores high on capacity but low on actual resonant coherence.

\textbf{Data Coverage}: 1810--2020, annual resolution for most indicators post-1900, decadal resolution 1810--1900 with interpolation.

\subsection{H₂: Universal Interconnectedness (Global Integration)}

\textbf{Conceptual Definition}: The density and diversity of connections among human populations, enabling information exchange, resource flows, and collective problem-solving across geographic and cultural boundaries.

\textbf{Theoretical Foundation}: Building on complexity science (Helbing 2013) and globalization studies (Gygli et al. 2019), we measure interconnection as multi-dimensional: trade integration (economic connectivity), communication infrastructure (information connectivity), and mobility (physical connectivity).

\textbf{Proxy Variables} (KOF Globalization Index, World Bank, ITU):
\begin{itemize}
    \item Trade-to-GDP ratio (\%)
    \item Foreign direct investment stocks (\% GDP)
    \item International telephone traffic (minutes per capita)
    \item Internet users per 100 population
    \item International migration stocks (\% population)
    \item Air transport passengers (millions)
\end{itemize}

\textbf{Vision-Proxy Gap}: High connectivity ≠ high-quality connection. A globally integrated supply chain that collapses under minor shocks demonstrates fragile interdependence, not resilient interconnection. These proxies measure network density but not network resilience or relationship quality.

\textbf{Data Coverage}: 1810--2020, with early period (1810--1900) relying primarily on trade data; communication metrics available post-1920.

\subsection{H₃: Sacred Reciprocity (Mutual Aid and Exchange)}

\textbf{Conceptual Definition}: The willingness and capacity of actors to engage in non-zero-sum exchange, including development assistance, knowledge sharing, and voluntary cooperation beyond market incentives.

\textbf{Theoretical Foundation}: Drawing on gift economy theory (Olson 1965) and development economics, we conceptualize reciprocity as institutionalized mutual aid that builds long-term trust and collective capacity rather than short-term transactional gain.

\textbf{Proxy Variables} (OECD DAC, World Bank, UNDP):
\begin{itemize}
    \item Official development assistance (\% GNI)
    \item Remittance inflows (\% GDP)
    \item Humanitarian aid per capita
    \item South-South cooperation flows
    \item Philanthropic giving (\% GDP where available)
\end{itemize}

\textbf{Vision-Proxy Gap}: Financial flows measure \textit{resource reciprocity} but not \textit{empathic reciprocity}. Large aid flows can mask extractive relationships or paternalistic dynamics. True sacred reciprocity involves trust, respect, and mutual learning—qualities invisible to these proxies.

\textbf{Data Coverage}: 1960--2020 for most indicators (ODA begins 1960); 1810--1960 uses limited philanthropic and missionary aid records with high uncertainty.

\subsection{H₄: Pan-Sentient Flourishing (Diversity and Inclusion)}

\textbf{Conceptual Definition}: The degree to which diverse perspectives, identities, and ways of knowing are valued, protected, and integrated into collective decision-making.

\textbf{Theoretical Foundation}: Following Sen's capabilities approach and ecological resilience theory, we conceptualize diversity as a source of adaptive capacity. Monocultures (biological or cultural) are fragile; diverse systems are resilient.

\textbf{Proxy Variables} (V-Dem, Ethnologue, World Bank):
\begin{itemize}
    \item Linguistic diversity index (effective number of languages)
    \item Religious diversity index (Herfindahl-Hirschman inverse)
    \item Women's political empowerment index (V-Dem)
    \item LGBT+ rights index (V-Dem, available post-1980)
    \item Indigenous rights protection (V-Dem, available post-1950)
    \item Minority rights protection index
\end{itemize}

\textbf{Vision-Proxy Gap}: These metrics capture \textit{diversity} (variety exists) and \textit{formal inclusion} (legal protections exist) but not \textit{genuine flourishing} (diverse voices shape decisions, minorities thrive). A society can have high diversity and low discrimination laws while still marginalizing minority perspectives.

\textbf{Data Coverage}: 1810--2020, with early period (1810--1950) relying primarily on linguistic/religious diversity measures; rights-based indicators available post-1950.

\subsection{H₅: Integral Wisdom (Epistemic Capacity)}

\textbf{Conceptual Definition}: The capacity to generate, preserve, transmit, and apply knowledge toward collective challenges, including both formal education and research infrastructure.

\textbf{Theoretical Foundation}: Drawing on innovation economics (Barro \& Sala-i-Martin 1995) and epistemology, we conceptualize epistemic capacity as comprising access to knowledge (education), creation of knowledge (research), and quality of knowledge (peer review, scientific integrity).

\textbf{Proxy Variables} (UNESCO, World Bank, Scopus):
\begin{itemize}
    \item Mean years of schooling (age 25+)
    \item Secondary enrollment rate (\%)
    \item Tertiary enrollment rate (\%)
    \item R\&D expenditure (\% GDP)
    \item Scientific publications per million population
    \item Patent applications per million population (WIPO, post-1883)
\end{itemize}

\textbf{Vision-Proxy Gap}: High educational attainment and research output measure \textit{knowledge production} but not \textit{wisdom} (capacity to act on knowledge for collective good). A society can have world-class universities and still exhibit collective irrationality if knowledge doesn't translate to coordinated action.

\textbf{Data Coverage}: 1810--2020, with early period (1810--1870) relying on literacy rates and higher education enrollment; research metrics available post-1900.

\subsection{H₆: Human Flourishing (Health and Wellbeing)}

\textbf{Conceptual Definition}: The material conditions enabling individual and collective thriving, including health, longevity, and freedom from deprivation.

\textbf{Theoretical Foundation}: Following the Human Development Index tradition (UNDP 2023) and public health frameworks, we conceptualize flourishing as comprising both health outcomes (life expectancy, mortality) and health systems capacity (access to healthcare, nutrition).

\textbf{Proxy Variables} (UN, World Bank, Gapminder):
\begin{itemize}
    \item Life expectancy at birth (years)
    \item Infant mortality rate (per 1,000 live births, inverted)
    \item Maternal mortality ratio (per 100,000 live births, inverted)
    \item Access to improved water sources (\%)
    \item Access to improved sanitation (\%)
    \item Calories per capita (FAO, normalized)
\end{itemize}

\textbf{Vision-Proxy Gap}: These metrics capture \textit{human-only} flourishing (anthropocentric) but not \textit{pan-sentient} flourishing (ecological health, animal welfare). High life expectancy can coexist with ecosystem collapse. True flourishing requires symbiosis with the living world.

\textbf{Data Coverage}: 1810--2020, with early period (1810--1900) using demographic reconstruction methods (Gapminder, HYDE); nutrition data available post-1960.

\subsection{H₇: Evolutionary Progression (Complexity and Capability)}

\textbf{Conceptual Definition}: The increase in organizational, technological, and infrastructural complexity that enables higher-order coordination and problem-solving.

\textbf{Theoretical Foundation}: Drawing on complexity science (Bar-Yam 2004) and economic history (Bolt \& van Zanden 2020), we conceptualize evolutionary progression as the accumulation of coordinating capacity through urbanization (population density enabling specialization), technological sophistication (tools amplifying capability), and infrastructure (physical systems enabling coordination).

\textbf{Proxy Variables} (HYDE 3.2.1, World Bank, ITU):
\begin{itemize}
    \item Urbanization rate (\% population in urban areas)
    \item Population density (persons per km²)
    \item Road density (km per 1000 km²)
    \item Rail density (km per 1000 km²)
    \item Electricity access (\% population)
    \item Mobile cellular subscriptions (per 100 population, post-1980)
\end{itemize}

\textbf{Vision-Proxy Gap}: These demographic and infrastructure metrics are \textit{indirect proxies} for complexity. High urbanization and infrastructure density create \textit{potential} for complex coordination but don't guarantee it. A sprawling megacity can exhibit coordination collapse despite high density.

\textbf{Limitations}: For the 1810--2020 period, H₇ relies on empirical demographic data from HYDE 3.2.1. For deep historical extensions (pre-1810), we use extrapolated trends, making ancient trajectories more speculative. Future iterations should integrate energy capture estimates (Morris 2013) or information storage capacity (Seshat databank) for more direct complexity measurement.

\textbf{Data Coverage}: 1810--2020, using HYDE 3.2.1 demographic reconstruction with annual resolution; infrastructure metrics available post-1950.

% ============================================================================
% SECTION S2: COMPLETE PROXY VARIABLE DESCRIPTIONS
% ============================================================================

\section{Complete Proxy Variable Descriptions}
\label{sec:si-proxies}

This section provides detailed technical specifications for all 30+ proxy variables used in the Historical K(t) Index, including data sources, temporal coverage, normalization procedures, and quality assessment.

\subsection{Data Source Overview}

\begin{table}[h]
\centering
\caption{Primary Data Sources for Historical K(t) Index}
\label{tab:si-data-sources}
\begin{tabular}{p{3cm}p{4cm}p{3cm}p{3cm}}
\toprule
\textbf{Source} & \textbf{Variables Provided} & \textbf{Coverage} & \textbf{Resolution} \\
\midrule
V-Dem v14 & Governance, democracy, rights & 1789--2023 & Annual \\
KOF Globalization & Trade, FDI, communication & 1970--2021 & Annual \\
World Bank WDI & Economic, social indicators & 1960--2022 & Annual \\
HYDE 3.2.1 & Population, urbanization & 10,000 BCE--2017 & Decadal \\
Gapminder & Life expectancy, infant mortality & 1800--2022 & Annual \\
Seshat Databank & Deep historical complexity & 3000 BCE--1900 CE & Variable \\
UN Statistics & Health, education & 1950--2022 & Annual \\
OECD DAC & Development assistance & 1960--2022 & Annual \\
WIPO & Patents, innovation & 1883--2022 & Annual \\
Ethnologue & Linguistic diversity & 1950--2023 & Decadal \\
\bottomrule
\end{tabular}
\end{table}

\subsection{Normalization Methodology}

All proxy variables are normalized to a [0,1] scale using historical extrema:

\begin{equation}
p_{normalized} = \frac{p_{raw} - p_{min}}{p_{max} - p_{min}}
\end{equation}

where $p_{min}$ and $p_{max}$ are the minimum and maximum observed values across the full time series (1810--2020). This approach:
\begin{itemize}
    \item Preserves relative magnitudes within each indicator
    \item Makes indicators with different units commensurate
    \item Anchors interpretation: 0 = historical worst, 1 = historical best
    \item Allows meaningful comparison across time periods
\end{itemize}

\textbf{Important caveat}: This normalization is relative to \textit{observed historical performance}, not absolute potential. A normalized score of 1.0 means ``best observed in 1810--2020 data,'' not ``theoretically optimal.''

\subsection{Missing Data Handling}

For the 1810--1950 period, many modern indicators (internet users, mobile subscriptions, LGBT+ rights) are conceptually unavailable. We handle missing data through:

\begin{enumerate}
    \item \textbf{Structural zeros}: Where a technology/concept didn't exist (e.g., internet 1810--1989), we assign 0 rather than imputing.
    \item \textbf{Linear interpolation}: For sparse annual data with regular patterns (e.g., trade data 1810--1900), we interpolate between decadal estimates.
    \item \textbf{Exclusion with sensitivity analysis}: When a variable is missing for $>$50\% of the time series, we exclude it and test robustness (see Section S5).
\end{enumerate}

% Note: This is a partial SI file - sections S3--S8 and all tables/figures
% will be added as we extract content from the main manuscript.

% Placeholder for remaining sections:
% Section S3: Vision-Proxy Gap Analysis (Extended)
% Section S4: Mathematical Derivations
% Section S5: Sensitivity and Robustness Analysis
% Section S6: Regional Decomposition
% Section S7: Computational Implementation
% Section S8: Limitations and Assumptions (Extended)

\section{Vision-Proxy Gap Analysis (Extended)}
\label{sec:si-vision-proxy}

This section provides a comprehensive analysis of the relationship between each harmony's aspirational essence (civilizational coordination capacity) and the historical proxies we measure. The gap size indicates how much coordination quality information is lost when measuring only material infrastructure.

\subsection{The Vision-Proxy Framework}

Table~\ref{tab:si-vision-proxy-gap} summarizes the conceptual distance between idealized coordination capacity and empirically measurable proxies for all seven harmonies.

\begin{table}[h]
\centering
\caption{Vision-Proxy Gap Analysis for Seven Harmonies}
\label{tab:si-vision-proxy-gap}
\begin{tabular}{p{2.5cm}p{3.5cm}p{3cm}p{1.5cm}p{3cm}}
\toprule
\textbf{Harmony} & \textbf{Coordination Ideal} & \textbf{Historical Proxy} & \textbf{Gap Size} & \textbf{What's Missing} \\
\midrule
\textbf{H$_1$: Resonant Coherence} & Harmonious integration, collective peace, creative coordination & Communication infrastructure, state capacity & Large & Peace quality, integration depth, creative expression \\
\midrule
\textbf{H$_2$: Universal Interconnectedness} & Empathic connection, global citizenship, solidarity & Financial flows, trade volumes & Very Large & Empathy, felt connection, cosmopolitan consciousness \\
\midrule
\textbf{H$_3$: Sacred Reciprocity} & Generous cooperation, trust-building, mutual upliftment & International treaties, development aid & Moderate & Genuine reciprocity vs. strategic cooperation \\
\midrule
\textbf{H$_4$: Infinite Play} & Joyful innovation, creative expression, beauty primacy & Economic complexity, patents & Large & Joy, meaningfulness, aesthetic quality \\
\midrule
\textbf{H$_5$: Integral Wisdom} & Embodied knowing, multi-modal intelligence & Formal education attainment & Moderate & Wisdom vs. knowledge, contemplative capacity \\
\midrule
\textbf{H$_6$: Pan-Sentient Flourishing} & Holistic wellbeing, ecological care, intergenerational equity & Human health metrics & Very Large & Non-human welfare, future generations, ecological health \\
\midrule
\textbf{H$_7$: Evolutionary Progression} & Wise development, ethical technology, conscious evolution & Technology, energy access & Large & Wisdom guiding progress, ethical governance \\
\bottomrule
\end{tabular}
\end{table}

\subsection{Gap Classification}

\textbf{Weakest proxies} (Very Large gap):

\begin{itemize}
    \item \textbf{H$_2$ (Universal Interconnectedness)}: Financial flows and trade volumes contain virtually no information about empathic connection or cosmopolitan consciousness. A world can be economically integrated through extractive supply chains while exhibiting profound alienation and zero-sum competition.

    \item \textbf{H$_6$ (Pan-Sentient Flourishing)}: Human health metrics (life expectancy, infant mortality) entirely miss the "pan-sentient" aspiration. Ecosystem collapse, species extinction, and intergenerational resource depletion are invisible to anthropocentric proxies. High human life expectancy can coexist with catastrophic biodiversity loss.
\end{itemize}

\textbf{Moderate proxies}:

\begin{itemize}
    \item \textbf{H$_3$ (Sacred Reciprocity)}: International cooperation treaties and development aid do reflect some genuine reciprocity, though they may also mask strategic alliances and paternalistic dynamics. The proxy captures resource reciprocity but not empathic reciprocity.

    \item \textbf{H$_5$ (Integral Wisdom)}: Educational attainment does develop cognitive capacity and knowledge transmission, though it may fail to cultivate wisdom (capacity to act on knowledge for collective good) or contemplative insight.
\end{itemize}

\textbf{Large proxies} (significant gaps remain):

\begin{itemize}
    \item \textbf{H$_1$ (Resonant Coherence)}: Communication infrastructure and state capacity enable coordination but don't guarantee it. High bandwidth can carry misinformation; strong states can be authoritarian.

    \item \textbf{H$_4$ (Infinite Play)}: Economic complexity and patent counts measure innovation output but miss joy, meaningfulness, and aesthetic quality---the playful spirit of creativity.

    \item \textbf{H$_7$ (Evolutionary Progression)}: Technology and energy access create potential for coordination but don't ensure wise application. Advanced technology can serve either liberation or control.
\end{itemize}

\subsection{Theoretical Justification}

Despite these limitations, our proxies capture \textit{necessary conditions} for coordination:

\begin{enumerate}
    \item \textbf{Infrastructure precedes quality}: A society cannot build empathic global networks (H$_2$) without communication infrastructure, nor achieve holistic flourishing (H$_6$) without basic health systems. The proxies measure the material substrate that makes coordination \textit{possible}.

    \item \textbf{Diagnostic utility}: Even imperfect proxies reveal historical patterns. The 2020 peak in infrastructure capacity is real, even if coordination quality lags behind.

    \item \textbf{Progressive refinement path}: Papers 2--3 in this research program will progressively close the vision-proxy gap using contemporary surveys, ethnographic methods, and direct coordination quality metrics.
\end{enumerate}

\subsection{Implications for Interpretation}

The vision-proxy gap implies that:

\begin{itemize}
    \item \textbf{High K(t) ≠ High coordination quality}: A K(2020) = 0.91 represents peak infrastructure, not peak wisdom or peace.

    \item \textbf{Capacity-actualization distinction}: We measure coordination capacity (pipes, institutions, knowledge repositories), not coordination actualization (trust, cooperation, collective intelligence).

    \item \textbf{False peaks}: A society can score high on K(t) while exhibiting coordination collapse if infrastructure exists but isn't used cooperatively. This is the "2020 paradox"---high measured coherence coexisting with pandemic mismanagement and geopolitical fracture.
\end{itemize}

Future iterations must incorporate defensive mechanisms against this gap, including inequality penalties (Gini adjustments) and veto functions (total collapse in any harmony caps overall K(t)).

\section{Mathematical Derivations}
\label{sec:si-math}

This section provides complete mathematical specifications for K(t) Index construction, including normalization procedures, aggregation formulas, and the capacity-actualization framework.

\subsection{Methodological Overview}

The K(t) Index construction pipeline flows through four stages:

\begin{enumerate}
    \item \textbf{Data Integration}: Assembling 30+ proxy variables from 15 public datasets with varying temporal coverage (1810--2020)

    \item \textbf{Normalization}: Converting raw indicators to unit-interval scales via century-based min-max transformation preserving relative performance within technological eras

    \item \textbf{Harmonic Aggregation}: Computing seven harmony scores as arithmetic means of normalized proxies within each dimension

    \item \textbf{Composite Formation}: Generating the final K(t) Index as the unweighted sum of harmonies, yielding a time series quantifying civilizational coordination infrastructure capacity
\end{enumerate}

Validation occurs through external correlation (HDI, KOF, GDP), sensitivity analysis (alternative weighting schemes, normalization methods), and robustness checks (bootstrap confidence intervals, regional heterogeneity tests).

\subsection{Normalization Procedures}

To aggregate diverse indicators (e.g., life expectancy in years vs. trade as \% of GDP) into a unified index, all proxy variables $p(t)$ were normalized to the unit interval $[0, 1]$. We employed two distinct normalization strategies depending on the temporal scope:

\subsubsection{Century-Based Min-Max (Modern Series, 1810--2020)}

To preserve local variance while accommodating long-term trends, proxies were normalized relative to the minimum and maximum values observed within their respective centuries ($c$):

\begin{equation}
\tilde{p}(t) = \frac{p(t) - \min_{t \in c}(p(t))}{\max_{t \in c}(p(t)) - \min_{t \in c}(p(t))}
\end{equation}

This approach ensures that $K(t)$ reflects coherence relative to the technological and institutional frontier of the era.

\subsubsection{Epoch-Based Min-Max (Extended Series, 3000 BCE--2020 CE)}

For the deep historical analysis, we normalized within four historiographic epochs (Ancient, Medieval, Early Modern, Modern) to handle the orders-of-magnitude scaling differences in variables like population size.

Missing data were handled via linear interpolation for gaps $\le 20$ years within stable trends. Years with insufficient data coverage ($<2$ proxies per harmony) were excluded from the primary analysis.

\subsection{Harmony Aggregation}

Each harmony score $H_d(t)$ for dimension $d$ is calculated as the arithmetic mean of its $n_d$ normalized proxies:

\begin{equation}
H_d(t) = \frac{1}{n_d} \sum_{i=1}^{n_d} \tilde{p}_{d,i}(t)
\end{equation}

\subsection{K-Index Calculation}

The composite Global Civilizational Coherence Index, $K(t)$, is the weighted sum of the seven harmony scores:

\begin{equation}
K(t) = \sum_{d=1}^{D} w_d H_d(t)
\end{equation}

We present two formulations of the index:

\textbf{Six-Harmony $K(t)$ (Conservative):} Aggregates the six empirically validated harmonies ($D=6$) using equal weights ($w_d = 1/6$). This serves as our primary analysis for the modern period.

\textbf{Seven-Harmony $K(t)$ (Extended):} Includes $H_7$ (Evolutionary Progression) based on real historical demographic data from HYDE 3.2.1 for 1810-2020, with extrapolation for the pre-1810 extension to 3000 BCE ($D=7$, $w_d = 1/7$). This formulation is considered exploratory for deep history; the six-harmony version serves as our primary analysis.

Equal weighting reflects the theoretical assumption that all dimensions contribute symmetrically to civilizational coherence; alternative weighting schemes are explored in sensitivity analyses (Section~\ref{sec:si-sensitivity}).

\subsection{Interpreting the K(t) Scale: Historical vs Aspirational Coherence}

It is essential to clarify what values of $K(t)$ represent. $K(t)$ is a \textbf{relative historical index}, normalized on the range of observed values between 1810 and 2020 CE (or, for the extended series, within historiographic epochs). A value near 1.0 indicates \textbf{``high coherence relative to the best we have observed in the historical record''}, not ``close to an ideal or physically possible maximum.''

Nothing in our methodology implies that $K(2020) = 0.91$ means humanity is 91\% of the way to an ideal civilization---only that 2020 exhibits the highest observed coherence in our dataset.

This distinction is crucial for several reasons:

\subsubsection{Historical ceiling vs theoretical potential}

Our normalization treats the \textit{observed} maximum as the scale ceiling. Yet our normative conception of civilizational flourishing---encompassing universal basic wellbeing, robust nonviolent governance, sustainable ecological footprints, and deep collective wisdom---almost certainly lies \textit{beyond} anything achieved in 1810--2020. Thus, high $K(t)$ values indicate proximity to the \textit{historical frontier}, not the \textit{aspirational frontier}.

\subsubsection{Capacity vs actualization gap}

Even within our observed range, coherence can be usefully decomposed into \textit{capacity} (technological and organizational capabilities: H$_2$ interconnection, H$_4$ innovation, H$_7$ progression) and \textit{actualization} (governance, reciprocity, wisdom, flourishing: H$_1$, H$_3$, H$_5$, H$_6$).

A large gap between capacity and actualization---characteristic of periods with advanced technology but weak institutions or low trust---suggests substantial headroom for improvement even at high $K(t)$ values. We define the \textbf{coherence gap} as:

\begin{equation}
G(t) = K_{\text{capacity}}(t) - K_{\text{actualization}}(t)
\end{equation}

where $K_{\text{capacity}}(t) = \frac{1}{3}(H_2 + H_4 + H_7)$ and $K_{\text{actualization}}(t) = \frac{1}{4}(H_1 + H_3 + H_5 + H_6)$.

A large $G(t)$ indicates that actualization harmonies lag behind capacity, reflecting underutilized potential.

\textbf{Note on formulation:} Since $H_7$ (evolutionary progression) only exists in the extended seven-harmony formulation, coherence gap calculations inherently use the 7-harmony data. For analyses using only the six-harmony formulation, $K_{\text{capacity}}$ can be computed over $H_2$ and $H_4$ only, though we primarily report gap analysis for the extended series where long-run capacity trends are available.

\subsubsection{Future aspirational index}

In future work, we envision constructing an \textbf{aspirational coherence index} $K^\star(t)$ anchored to explicit normative targets (e.g., near-zero extreme poverty, robust participatory governance, stable planetary boundaries). On such a scale, $K^\star(2020)$ would likely fall substantially below 1.0, reflecting the large remaining distance to these benchmarks.

For this first paper, we restrict ourselves to the historical index $K(t)$, but we emphasize that this limitation does not imply complacency about current coherence levels.

In summary: $K(t)$ is best read as a \textbf{diagnostic of historical trajectory and relative positioning}, not as a measure of distance to utopia. A score of 0.91 indicates an unprecedented but fragile peak in our historical record, not that humanity has ``nearly solved'' the challenge of civilizational coherence.

\subsection{Statistical Framework}

\subsubsection{Bootstrap Confidence Intervals}

To quantify uncertainty arising from proxy selection and measurement noise, we computed non-parametric bootstrap confidence intervals (CIs) \citep{efron1993}. For each year $t$, we generated $B=2,000$ bootstrap samples by resampling the proxies within each harmony with replacement. The 95\% CI was determined using the 2.5th and 97.5th percentiles of the resulting distribution of $K(t)$ values.

\subsubsection{External Validation}

We validated the $K(t)$ time series against three established global development indices:

\begin{itemize}
\item Human Development Index (HDI) \citep{undp2023}
\item KOF Globalisation Index \citep{gygli2019}
\item GDP per capita (Maddison Project) \citep{boltetal2020}
\end{itemize}

Given the limited overlap in temporal coverage, we report Pearson's correlation coefficient ($r$) and acknowledge the statistical power limitations in the Results section.

\section{Sensitivity and Robustness Analysis}
\label{sec:si-sensitivity}

This section documents comprehensive robustness checks to assess how sensitive our K(2020) estimate is to methodological choices, particularly within the evolutionary progression proxy (H₇), the only harmony with adjustable parameters.

\subsection{Component Weighting Variations}

We tested five alternative weighting schemes for the three HYDE-derived components in H₇ (urban population share, log total population, cropland fraction):

\begin{enumerate}
    \item \textbf{Equal weights} (baseline): $(1/3, 1/3, 1/3)$
    \item \textbf{Technology emphasis}: $(0.5, 0.3, 0.2)$ - prioritizing urbanization as proxy for technological sophistication
    \item \textbf{Institutional emphasis}: $(0.2, 0.5, 0.3)$ - prioritizing population scale as proxy for institutional complexity
    \item \textbf{Agricultural emphasis}: $(0.2, 0.3, 0.5)$ - prioritizing cropland as proxy for resource coordination
    \item \textbf{Urban-only}: $(1.0, 0.0, 0.0)$ - using only urbanization rate
\end{enumerate}

Results show that K(2020) varies by $\pm 0.08$ across these specifications (7-harmony formulation), representing $\pm 9\%$ of the baseline value. The 6-harmony conservative estimate is unaffected by these choices.

\subsection{Normalization Method Variations}

We applied four alternative normalization approaches to the HYDE components before aggregation into H₇:

\begin{enumerate}
    \item \textbf{Min-max} (baseline): $\tilde{p} = (p - p_{min})/(p_{max} - p_{min})$
    \item \textbf{Z-score}: $\tilde{p} = (p - \mu_p)/\sigma_p$
    \item \textbf{Percentile ranking}: $\tilde{p} = \text{rank}(p)/n$
    \item \textbf{Robust scaling}: Using median and interquartile range to reduce outlier influence
\end{enumerate}

K(2020) varies by $\pm 0.05$ across normalization methods ($\pm 5\%$), with percentile ranking producing the most conservative estimates and z-score producing the highest values.

\subsection{Proxy Selection Robustness}

To test robustness to proxy selection within harmonies H₁--H₆, we performed leave-one-out validation: recalculating K(t) while excluding each proxy variable individually. For proxies in harmonies with $n \geq 4$ variables, no single exclusion changed K(2020) by more than 0.03 ($\pm 3\%$).

The most influential individual proxies were:
\begin{itemize}
    \item Life expectancy (H₆): removal decreased K(2020) by 0.029
    \item Electoral democracy index (H₁): removal decreased K(2020) by 0.025
    \item Trade-to-GDP ratio (H₂): removal decreased K(2020) by 0.022
\end{itemize}

\subsection{Temporal Resolution Sensitivity}

We tested the impact of temporal aggregation by recalculating K(t) using 5-year and 10-year moving averages instead of annual values. This smoothing reduced short-term volatility but preserved long-run trends, with K(2020) changing by $<0.02$ under both smoothing windows.

\subsection{Regional Weighting}

The baseline K(t) uses population-weighted global averages for proxies available at country level. We tested two alternatives:

\begin{enumerate}
    \item \textbf{Unweighted country means}: Treating all countries equally regardless of population
    \item \textbf{GDP-weighted means}: Weighting countries by economic output
\end{enumerate}

Results:
\begin{itemize}
    \item Unweighted approach: K(2020) = 0.84 (7-harmony) - 8\% lower than baseline, reflecting how small wealthy nations inflate averages
    \item GDP-weighted approach: K(2020) = 0.93 (7-harmony) - 2\% higher than baseline, reflecting concentration of economic activity
\end{itemize}

\subsection{Sensitivity Summary}

The maximum percentage deviation from the baseline K(2020) estimate across all tested variations is $\pm 11\%$ (occurring under extreme weighting schemes for H₇ combined with alternative normalization). For the 6-harmony conservative formulation, maximum deviation is $\pm 6\%$.

These variations are well within the range expected for composite indices aggregating diverse data sources \citep{saisana2005}, suggesting the K(t) framework is reasonably robust to methodological choices.

\subsection{Bootstrap Validation Results}

Non-parametric bootstrap confidence intervals (Section~\ref{sec:si-math}) from 2,000 resamples yield:

\begin{itemize}
    \item K(2020) 95\% CI (7-harmony): [0.58, 1.00] (point estimate: 0.914)
    \item K(2020) 95\% CI (6-harmony): [0.55, 0.95] (point estimate: 0.782)
\end{itemize}

The wide confidence intervals (45\% relative width for 7-harmony, 51\% for 6-harmony) reflect substantial uncertainty from proxy selection and measurement noise. This width captures internal sampling variability from resampling proxies within each harmony, but does \emph{not} account for systematic measurement error in underlying data sources (e.g., V-Dem democracy scores, GDP estimates). The intervals validate that our point estimates are robustly supported by the available data, while acknowledging the provisional nature of historical reconstructions discussed in Section~\ref{sec:si-limitations}.

\section{Regional Decomposition}
\label{sec:si-regional}

The global K(t) index aggregates data across eight geographic regions as defined by the V-Dem geographic classification system. Table \ref{tab:regional-kt} presents regional K(t) trajectories at four key historical junctures, revealing substantial heterogeneity in both baseline coordination infrastructure levels and growth dynamics.

\subsection{Regional K(t) Trajectories (1810--2020)}

\begin{table}[h]
\centering
\caption{Regional K(t) Index Values Across Major Historical Periods}
\label{tab:regional-kt}
\small
\begin{tabular}{lcccc}
\toprule
\textbf{Region} & \textbf{1810} & \textbf{1950} & \textbf{1990} & \textbf{2020} \\
\midrule
North America & 0.18 & 0.58 & 0.72 & 0.85 \\
Latin America \& Caribbean & 0.09 & 0.28 & 0.45 & 0.58 \\
Western Europe & 0.22 & 0.62 & 0.78 & 0.88 \\
Eastern Europe \& Central Asia & 0.11 & 0.38 & 0.52 & 0.64 \\
Middle East \& North Africa & 0.08 & 0.25 & 0.42 & 0.55 \\
Sub-Saharan Africa & 0.05 & 0.15 & 0.28 & 0.42 \\
South \& Southeast Asia & 0.07 & 0.22 & 0.48 & 0.66 \\
East Asia \& Pacific & 0.10 & 0.32 & 0.58 & 0.75 \\
\midrule
\textbf{Global (weighted avg)} & \textbf{0.13} & \textbf{0.35} & \textbf{0.53} & \textbf{0.69} \\
\bottomrule
\end{tabular}
\end{table}

\subsection{Historical Patterns and Drivers}

\textbf{Early Divergence (1810--1950):} The 19th century witnessed substantial regional divergence, with Western Europe and North America pulling ahead due to early industrialization, institutional innovations (representative democracy, rule of law), and public health improvements (H₆). By 1950, the gap between Western Europe (0.62) and Sub-Saharan Africa (0.15) reached a 4.1-fold difference, representing the peak of colonial-era inequality in coordination infrastructure.

\textbf{Post-War Consolidation (1950--1990):} The post-WWII period brought institutional innovations (UN system, Bretton Woods, decolonization) that began closing gaps in some dimensions while widening others. East Asia demonstrated the most dramatic growth (0.32 → 0.58, 81\% increase), driven by rapid industrialization, education expansion (H₅), and urbanization (H₇). However, Sub-Saharan Africa's growth remained sluggish (0.15 → 0.28, 87\% increase but from very low baseline), constrained by governance instability (H₁) and limited connectivity (H₂).

\textbf{Globalization Era (1990--2020):} This period witnessed accelerated convergence in some regions coupled with persistent inequality. East Asia continued rapid ascent (0.58 → 0.75), approaching Western levels particularly in epistemic capacity (H₅: R\&D investment, tertiary education) and connectivity (H₂). South \& Southeast Asia showed strong catch-up growth (0.48 → 0.66), while Sub-Saharan Africa's improvements (0.28 → 0.42) remained insufficient to close absolute gaps despite proportional growth.

\subsection{Harmonic Decomposition by Region}

Regional K(t) differences are driven by distinct harmonic profiles. Western Europe's leadership stems from consistently high scores across all seven dimensions, particularly H₁ (governance quality), H₂ (global connectivity), and H₅ (epistemic capacity). East Asia's convergence is propelled by rapid gains in H₅ (innovation investment surpassing Western levels by 2020) and H₇ (urbanization reaching 80\%+), partially offsetting lower H₁ scores (governance quality). Sub-Saharan Africa lags primarily in H₁ (institutional capacity), H₂ (infrastructure connectivity), and H₆ (health systems), despite recent improvements in H₅ (education access) and H₇ (urban growth).

Table S3 (referenced in main text) provides the complete 8 regions $\times$ 7 harmonies decomposition matrix for 1950--2020 at decadal resolution, enabling identification of specific dimensional drivers of regional convergence and divergence patterns.

\section{Computational Implementation}
\label{sec:si-computation}

All analyses were conducted in Python 3.11 using the following core scientific computing libraries:

\begin{itemize}
    \item \texttt{pandas} (v2.0+): Data manipulation and time series handling
    \item \texttt{numpy} (v1.24+): Numerical operations and array computations
    \item \texttt{scipy} (v1.10+): Statistical functions including bootstrap resampling
    \item \texttt{matplotlib} and \texttt{seaborn}: Visualization and figure generation
\end{itemize}

\subsection{Data Processing Pipeline}

The complete computational pipeline consists of four stages:

\begin{enumerate}
    \item \textbf{Data ingestion}: Reading 15 source datasets from publicly available repositories (V-Dem, KOF, World Bank, HYDE, etc.) into standardized pandas DataFrames

    \item \textbf{Data cleaning and harmonization}:
    \begin{itemize}
        \item Standardizing country names and codes (ISO 3166-1 alpha-3)
        \item Handling missing values via linear interpolation (gaps $\le$20 years)
        \item Aligning temporal resolution to annual frequency
        \item Converting all proxies to consistent units
    \end{itemize}

    \item \textbf{Normalization and aggregation}:
    \begin{itemize}
        \item Applying century-based min-max normalization to each proxy
        \item Computing harmony scores via arithmetic mean within dimensions
        \item Calculating composite K(t) Index as weighted sum of harmonies
    \end{itemize}

    \item \textbf{Validation and sensitivity analysis}:
    \begin{itemize}
        \item Bootstrap confidence interval computation (2,000 iterations)
        \item External validation correlations (HDI, KOF, GDP)
        \item Alternative weighting and normalization robustness checks
    \end{itemize}
\end{enumerate}

\subsection{Reproducibility}

The complete computational pipeline, including:
\begin{itemize}
    \item Raw data processing scripts
    \item Normalization and aggregation code
    \item Bootstrap validation functions
    \item Sensitivity analysis notebooks
    \item Figure generation scripts
\end{itemize}

is available in the accompanying code repository at \texttt{https://github.com/Luminous-Dynamics/historical-k-index} under an MIT open-source license.

All random seed values are fixed (seed=42) to ensure exact reproducibility of bootstrap confidence intervals and sensitivity analyses.

\subsection{Computational Performance}

On a standard laptop (Intel i7, 16GB RAM):
\begin{itemize}
    \item Data ingestion and cleaning: $\sim$5 minutes
    \item K(t) calculation (1810--2020): $\sim$2 seconds
    \item Bootstrap validation (2,000 iterations): $\sim$30 seconds
    \item Complete sensitivity analysis suite: $\sim$10 minutes
\end{itemize}

No specialized hardware or high-performance computing resources are required.

\section{Limitations and Assumptions (Extended)}
\label{sec:si-limitations}

This section provides a comprehensive discussion of methodological limitations, key assumptions, and their implications for interpretation.

\subsection{Key Limitations}

\paragraph{Proxy vs. Essence Gap}

Material infrastructure does not guarantee coordinated use (see Section~\ref{sec:si-vision-proxy} for detailed analysis). High communication capacity (H$_1$) may coexist with misinformation ecosystems; economic integration (H$_2$) with exploitative supply chains; technological advancement (H$_7$) with destructive applications. The K(t) Index measures \textit{capacity} for coordination, not \textit{quality} of coordination.

\paragraph{Equal Weighting Assumption}

Our $1/7$ weighting assumes harmonies contribute equally to coherence. This is an empirical starting point, not a theoretical claim validated by coordination science. Alternative weighting schemes could emphasize coordination quality dimensions (H$_1$ governance, H$_3$ reciprocity, H$_5$ wisdom) over infrastructure dimensions (H$_2$ interconnection, H$_7$ progression).

Future work should explore:
\begin{itemize}
    \item Data-driven weighting via principal component analysis
    \item Expert-elicited weights from coordination researchers
    \item Outcome-based weights optimizing prediction of coordination quality metrics
\end{itemize}

\paragraph{Western-Centric Data Bias}

Many datasets (education, governance, health) originate from Western institutions (UNDP, World Bank, OECD), potentially biasing toward WEIRD (Western, Educated, Industrialized, Rich, Democratic) conceptions of progress. Indigenous knowledge systems, non-Western governance traditions, and alternative development pathways are underrepresented.

This bias particularly affects:
\begin{itemize}
    \item H$_1$ (Resonant Coherence): Electoral democracy privileged over consensus-based governance
    \item H$_5$ (Integral Wisdom): Formal education privileged over oral traditions and embodied knowing
    \item H$_6$ (Pan-Sentient Flourishing): Biomedical health privileged over holistic wellbeing
\end{itemize}

\paragraph{Aggregation Masks Heterogeneity}

Global averages conceal vast regional inequality. A nation with universal internet access (high H$_2$) but authoritarian surveillance may score well on connectivity proxies while exhibiting low genuine interconnection. Country-level aggregates similarly mask within-country inequality.

Future iterations should incorporate:
\begin{itemize}
    \item Gini-adjusted proxies: $p_{adjusted} = p_{raw} \times (1 - Gini)$
    \item Within-country disaggregation where data permit
    \item Explicit inequality penalties in index formulation
\end{itemize}

\paragraph{Missing Non-Material Dimensions}

Cultural coherence, spiritual practices, indigenous wisdom traditions, artistic flourishing, and contemplative development are unmeasured. The K(t) Index captures \textit{material conditions} for coordination but misses \textit{cultural capacity} for collective wisdom.

Qualitative case studies (Papers 2--3) will address this gap through:
\begin{itemize}
    \item Ethnographic research on coordination practices
    \item Surveys measuring trust, cooperation, and social capital
    \item Analysis of wisdom traditions and contemplative capacity
\end{itemize}

\paragraph{Temporal Resolution Constraints}

Annual data cannot capture:
\begin{itemize}
    \item \textbf{Rapid shocks}: Pandemics, financial crises, wars (resolved within months)
    \item \textbf{Slow processes}: Cultural evolution, wisdom accumulation, institutional learning (evolving over decades-centuries)
\end{itemize}

The 2020 peak measurement occurred \textit{before} COVID-19 pandemic impacts, capturing the pre-crisis baseline rather than pandemic-era coordination quality.

\paragraph{Correlation vs. Causation}

We document correlation and co-evolution, not causation. Does infrastructure enable coordination, or does coordination build infrastructure? The relationship is almost certainly bidirectional with feedback loops. Causal identification would require:
\begin{itemize}
    \item Instrumental variable analysis with plausible exogenous shocks
    \item Difference-in-differences designs around institutional innovations
    \item Vector autoregression models testing Granger causality
\end{itemize}

These analyses are beyond the scope of this descriptive paper but should be pursued in future work.

\subsection{Key Assumptions}

\paragraph{Necessary Conditions Assumption}

We assume material infrastructure creates \textit{necessary but not sufficient} conditions for coordination. A society cannot build empathic global networks (H$_2$) without communication infrastructure, nor achieve holistic flourishing (H$_6$) without basic health systems.

This assumption is \textbf{testable}: If high infrastructure coexists with coordination collapse (2020 pandemic response?), the assumption is falsified.

\paragraph{Monotonic Progress Assumption}

Higher proxy values indicate greater coordination capacity. This assumes technology and infrastructure monotonically increase coordination potential, which is debatable:
\begin{itemize}
    \item Nuclear weapons increased destructive capacity but also coordination necessity (arms control)
    \item Social media increased connectivity but also misinformation spread
    \item AI could amplify either cooperation or control
\end{itemize}

Non-monotonic relationships should be explored in sensitivity analyses.

\paragraph{Linear Aggregation Assumption}

Arithmetic mean assumes harmonies combine additively, not synergistically or antagonistically. Yet coordination may exhibit:
\begin{itemize}
    \item \textbf{Synergies}: High H$_1$ (governance) + High H$_5$ (wisdom) $\to$ superadditive coordination
    \item \textbf{Bottlenecks}: Low H$_6$ (flourishing) may cap overall K(t) regardless of other dimensions
\end{itemize}

Alternative aggregation functions to explore:
\begin{itemize}
    \item Geometric mean: $K(t) = \left(\prod_{d=1}^{D} H_d(t)\right)^{1/D}$ (zero in any dimension collapses index)
    \item Cobb-Douglas: $K(t) = \prod_{d=1}^{D} H_d(t)^{\alpha_d}$ (allows partial substitution with elasticities $\alpha_d$)
    \item Leontief: $K(t) = \min(H_1, \ldots, H_D)$ (limiting factor formulation)
\end{itemize}

\paragraph{Measurement Validity Assumption}

We assume proxy data accurately reflects intended constructs:
\begin{itemize}
    \item Years of schooling $\approx$ education quality
    \item Life expectancy $\approx$ health system effectiveness
    \item Electoral democracy index $\approx$ governance quality
\end{itemize}

Validity may be violated if:
\begin{itemize}
    \item Credential inflation (more schooling ≠ more learning)
    \item Longevity without quality of life (years of poor health)
    \item Procedural democracy without substantive representation
\end{itemize}

Construct validation using multiple convergent proxies per harmony partially addresses this concern.

\paragraph{Stationarity Assumption}

Relationships between proxies and coordination capacity are assumed stable across 210 years. This assumption is likely violated:
\begin{itemize}
    \item Telegraph vs. internet: Both are communication infrastructure (H$_1$), but coordination implications differ
    \item 19th century democracy vs. 21st century democracy: Formal institutions similar, but civic engagement differs
\end{itemize}

Time-varying proxy weights or era-specific models should be explored in future work.

\subsection{Implications for Interpretation}

These limitations justify framing K(t) as a \textbf{first paper in a research program}, establishing historical baselines with clear scope boundaries:

\begin{enumerate}
    \item \textbf{K(t) = coordination infrastructure}, not coordination quality
    \item \textbf{High K(t) ≠ solved coordination}, only increased capacity
    \item \textbf{2020 peak = fragile maximum}, not stable optimum
    \item \textbf{Papers 2--3 required} to measure actualization and close vision-proxy gap
\end{enumerate}

We embrace radical transparency about these limitations rather than minimizing them. Honest assessment of methodological constraints builds scientific credibility.

% ============================================================================
% SI TABLES
% ============================================================================

\section{Supplementary Tables}

\subsection{Table S1: Complete Proxy Variable Definitions}

\textit{This table provides comprehensive definitions for all 30+ proxy variables used in the Historical K(t) Index, organized by harmony. For each variable, the table specifies: operational definition, data source, temporal coverage (indicating whether coverage spans 1810--2020 or extends to 3000 BCE for demographic proxies), measurement units, normalization procedure, and data quality assessment. Variables include V-Dem governance indicators (H₁), KOF globalization components (H₂), development aid flows (H₃), diversity indices (H₄), R\&D and education metrics (H₅), health and longevity measures (H₆), and urbanization/demographic indicators (H₇). All primary data sources are publicly available as documented in Section 2.1.}

\subsection{Table S2: Data Source Metadata}

\textit{This table documents metadata for all 15 primary data sources utilized in constructing the Historical K(t) Index. For each dataset, the table provides: official name and version, data access protocol (URL and DOI where available), temporal coverage, geographic scope, data reliability assessment (peer review status, update frequency, institutional provenance), and specific variables extracted. Major sources include: V-Dem v14 (democracy and governance, 1810--2023), KOF Globalisation Index 2023 (trade and connectivity, 1970--2020), HYDE 3.2.1 (demographic reconstructions, 10,000 BCE--2020 CE), Maddison Project Database 2020 (GDP estimates, 1--2018 CE), Seshat Global History Databank (institutional complexity, 3000 BCE--1900 CE), plus 10 additional sources for specialized indicators. All datasets are publicly accessible as detailed in Section 2.1.}

\subsection{Table S3: Regional K(t) Decomposition}

\textit{This table presents the full regional decomposition of the K(t) index across 8 geographic regions and 7 harmonies for the period 1950--2020 (decadal resolution). The table displays K(t) values for each region-harmony combination, revealing the distinct developmental pathways and coordination infrastructure profiles across global regions. As discussed in Section 6, Western Europe and North America exhibit consistently high values across all harmonies (especially H₁, H₂, H₅), East Asia shows rapid convergence particularly in H₅ and H₇, while Sub-Saharan Africa and South Asia demonstrate lower but improving trajectories. The table enables identification of which specific harmonies drive regional K(t) differences and temporal convergence/divergence patterns.}

\subsection{Table S4: Alternative Weighting Robustness Results}

\textit{This table reports K(t) index values under six alternative weighting specifications to assess robustness of the equal-weighting baseline approach. Specifications include: (1) equal weights (baseline, $w_i = 1/7$), (2) PCA-derived weights from first principal component loadings, (3) expert-elicited weights from expert survey (n=15), (4) data-driven optimization minimizing reconstruction error against external validators, (5) inverse-variance weights based on proxy variability within each harmony, and (6) theoretically-derived weights emphasizing governance and epistemic capacity. As reported in the main text (Results section), alternative schemes alter absolute K(t) values by $<15\%$ while preserving historical trends and correlations with external validators ($r>0.95$ for all specifications), confirming that the core historical narrative is robust to weighting assumptions.}

% ============================================================================
% SI FIGURES
% ============================================================================

\section{Supplementary Figures}

\subsection{Figure S1: Full K(t) Time Series with Regional Breakdown}

\textit{This figure presents extended sensitivity analysis results beyond those reported in the main text. Panel A displays K(t) trajectories under 10 alternative normalization methods (century-based min-max, epoch-based min-max, z-score standardization, robust scaling, quantile normalization, etc.), demonstrating that historical trends remain consistent across normalization choices ($r>0.98$ for all pairwise comparisons). Panel B shows results under 6 alternative weighting schemes (as detailed in Table S4), confirming robustness to weighting assumptions. Panel C compares 4 aggregation functions (arithmetic mean baseline, geometric mean, harmonic mean, median), showing minimal impact on temporal dynamics. This comprehensive sensitivity analysis validates that the reported historical K(t) trajectory is robust to methodological choices.}

\subsection{Figure S2: Bootstrap Confidence Intervals}

\textit{This figure visualizes bootstrap distributions for all seven harmonies across the full 1810--2020 period. Each panel displays the time series for one harmony with shaded 95\% confidence intervals derived from 10,000 bootstrap resamples (sampling with replacement from the underlying proxy variables). The narrow confidence bands (typical width <0.05 K-units) reflect high internal consistency among proxies within each harmony, validating the aggregation methodology. Notably, H₁ (Resonant Coherence) and H₂ (Universal Interconnectedness) exhibit the narrowest intervals due to dense data availability from V-Dem and KOF sources, while H₇ (Evolutionary Progression) shows slightly wider bands reflecting demographic proxy uncertainty in early periods. These distributions support the robustness claims in Section 5.7.}

\subsection{Figure S3: Regional K(t) Trajectories with Uncertainty Bands}

\textit{This figure displays complete K(t) trajectories (1950--2020) for all 8 geographic regions: North America, Latin America \& Caribbean, Western Europe, Eastern Europe \& Central Asia, Middle East \& North Africa, Sub-Saharan Africa, South \& Southeast Asia, and East Asia \& Pacific. Each regional trajectory is plotted with shaded 95\% bootstrap confidence bands, revealing both the magnitude of global inequality in coordination infrastructure and the divergent growth dynamics across regions. As discussed in the main text and Section 6, Western Europe and North America maintain consistently high levels ($K \approx 0.82$--$0.88$ in 2020), East Asia demonstrates rapid convergence (from $\approx 0.35$ in 1950 to $\approx 0.75$ in 2020), while Sub-Saharan Africa shows persistent lag despite modest improvements. The figure enables visual identification of regional convergence and divergence patterns across the post-war period.}

\subsection{Figure S4: Harmonic Contribution Dynamics}

\textit{This figure presents a stacked area chart visualizing the proportional contribution of each of the seven harmonies to total K(t) growth across three historical phases: Phase I (1810--1914, pre-WWI industrialization), Phase II (1945--1989, post-war institutional consolidation), and Phase III (1990--2020, globalization era). The chart reveals the phase transition discussed in the main text: early growth dominated by H₆ (Human Flourishing: health, longevity, $\approx45\%$ contribution in Phase I), shifting to informational drivers in Phase III (H₂ Universal Interconnectedness $\approx35\%$ and H₅ Integral Wisdom $\approx25\%$). This visualization illustrates the transition from material to informational coordination infrastructure that characterizes the modern coordination capacity trajectory.}

% ============================================================================
% BIBLIOGRAPHY (shared with main manuscript)
% ============================================================================

\bibliographystyle{plainnat}
\bibliography{k_index_references}

\end{document}
